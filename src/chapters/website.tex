\documentclass{subfiles}

\begin{document}

\chapter{Webseite}\label{ch:website}
Die Webseite ist das Herzstück der CodeUp Plattform.
Sie ist der primäre Zugangspunkt für die Nutzer und bietet alle Funktionen, die für das Lernen und Programmieren benötigt werden.
In diesem Kapitel wird die Entwicklung der Webseite von den Anfängen bis zum aktuellen Stand beschrieben.

\section{Technische Hintergründe}\label{sec:website-tech}
Die Webseite ist als Multi-Page-Application (MPA) konzipiert, was bedeutet, dass sie aus mehreren einzelnen Seiten besteht, die über Links miteinander verbunden sind.
Realisiert wird dies mit dem JavaScript-Framework Next.js\footnote{\bscite{nextjs}}, das auf React\footnote{\bscite{react}} basiert und eine serverseitige Generierung von Webseiten ermöglicht.
Dadurch wird eine schnelle Ladezeit und eine gute Suchmaschinenoptimierung (SEO) erreicht, was für eine Lernplattform besonders wichtig ist.
Durch die Komplexität der Plattform und die Vielzahl von unterstützten Protokollen (z.B. HTTP, WebSockets, etc.) wurde ein eigener HTTP-Server implementiert,
der auf dem Node.js\footnote{\bscite{nodejs}} \dq http\dq-Modul basiert.
Dieser Server stellt zum Großteil nur eine Verbindung zur Next.js Bibliothek her und leitet Anfragen entsprechend weiter.
Neben dieser Funktionalität bietet er eine WebSocket Schnittstelle mit einem Socket.IO-Server\footnote{\bscite{socketio}} an, die für jegliche Echtzeitkommunikation verwendet wird.
Entwickelt wurden alle Komponenten der Webseite mit TypeScript\footnote{\bscite{typescript}}, um die Codequalität und -sicherheit zu erhöhen.
Beim Start der Anwendung wird eine Verbindung zu einem MongoDB\footnote{\bscite{mongodb}}-Server hergestellt, um Daten zu speichern und abzurufen.
Die Wahl viel auf MongoDB, da eine No-SQL-Datenbank für die Anwendung besser geeignet ist, als eine relationale Datenbank.
Durch die Verwendung einer solcne No-SQL-Datenbank können die Daten flexibler gespeichet werden, als es bei der klassischen tabellenbasierten Speicherung der Fall wäre,
da jedes Dokument in der Datenbank individuell gestaltet werden kann und nicht zwingend einem festen Schema folgen muss.
Des Weiteren wird durch eine No-SQL-Datenbank eine höhere Sicherheit gewährleistet, da MongoDB keine SQL-Injections zulässt, da es keine SQL-Abfragen gibt.
Der Quellcode der Webseite ist in einer Turborepo\footnote{\bscite{turborepo}}-Struktur organisiert, was bedeutet, dass alle Komponenten in einem einzigen Repository liegen
und so einfacher miteinander interagieren können.
Diese Komponenten umfassen unter anderen Komponenten für die Benutzeroberfläche, die Kommunikation mit dem Server, die Verwaltung der Datenbank und die Integration von Drittanbieterdiensten.
Für die Speicherung von Dateien wird ein MinIO\footnote{\bscite{minio}}-Object-Storage-Server verwendet, der eine einfache und effiziente Möglichkeit bietet, Dateien zu speichern und abzurufen.
Durch diesen Objekt-Storage können programmatisch Dateien hochgeladen und heruntergeladen werden, ohne dass ein direkter Zugriff auf das Dateisystem des Servers erforderlich ist.
Details zur Bereitstellung und Skalierung der Anwendung werden im Kapitel \ref{sec:infra-deployment} beschrieben.


\section{Benutzerkonten und Sicherheit}\label{sec:website-security}
CodeUp bietet die Möglichkeit, sich für ein Benutzerkonto zu registrieren, um alle Funktionen der Plattform nutzen zu können.
Dabei wird ein Benutzername, eine E-Mail-Adresse und ein Passwort benötigt.
Zusätzlich muss jeder Benutzer noch seinen Vor- und Nachnamen angeben, um die persönliche Ansprache zu ermöglichen.
Um die Sicherheit der Benutzerkonten zu gewährleisten, werden die Passwörter mit einer starken Hashfunktion gehasht und nie im Klartext gespeichert.
Es wird ein sha256\footnote{\bscite{sha256}}-Algorithmus verwendet, da dieser als sicher gilt.
Um eine zusätzliche Sicherheitsstufe zu erreichen, können Benutzer optional die Zwei-Faktor-Authentifizierung (2FA) aktivieren.
Dabei wird ein zusätzlicher Code benötigt, der über eine Authentifizierungs-App generiert wird.
Nur bei der korrekten Eingabe von Passwort und Code wird der Zugriff auf das Benutzerkonto gewährt.
Um die Sicherheit der Benutzerkonten weiter zu erhöhen, wird bei jedem Login-Versuch eine E-Mail an den Benutzer gesendet, um ihn über den Login zu informieren.
Bei einer erfolgreichen Anmeldung wird ein JSON-Web-Token (JWT)\footnote{\bscite{jwt-rfc}} generiert, der als Authentifizierungstoken dient.
Ein solcher JWT besteht aus drei Teilen: einem Header, einem Payload und einer Signatur.
Durch die Singatur wird sichergestellt, dass der Token nicht manipuliert wurde und nur von der CodeUp-Plattform generiert wurde.
Der Token ist in der Standardkonfiguration nur 30 Tage gültig und erfordert danach eine erneute Anmeldung.
Bei einer Anmeldung wird der generierte Token im LocalStorage des Browsers gespeichert und bei jder zukünftigen Anfrage an den Server mitgesendet.
Dadurch wird sichergestellt, dass der Benutzer eingeloggt bleibt, auch wenn die Seite neu geladen wird.
Um die Sicherheit der Benutzerdaten zu gewährleisten, wird die gesamte Kommunikation zwischen Client und Server über HTTPS verschlüsselt.

\section{Funktionsumfang}\label{sec:website-features}

\subsection{Generelle und globale Funktionen}

\subsubsection{Navigation}
Die Navigation ist das zentrale Element der Webseite und ermöglicht es den Benutzern, sich schnell und einfach zurechtzufinden.
Sie besteht aus einer Navigationsleiste am oberen Bildschirmrand und einer Fußzeile am Ende der Seite.
In der Navigationsleiste sind alle wichtigen Funktionen kategorisiert und über Links erreichbar (z.B. Startseite, Kurse, Profil, Editor, etc.).
Die Fußzeile enthält Einstellungen bezüglich der Sprache und weiter Informationen über die Plattform.

\subsubsection{Sprachauswahl}
Um jedem Benutzer die Möglichkeit zu geben, die CodeUp nutzen zu können, wird die Webseite in mehreren Sprachen angeboten.
Hierbei ist Deutsch die Standardsprache und bietet die besten Übersetzungen.
Alle anderen Sprachen (Englisch, Französisch, Spanisch, Italienisch, Niederländisch, Polnisch und Portugiesisch) sind maschinell übersetzt und können deswegen teilweise ungenau sein.
Die Sprache wird über ein Fragment der URL festgelegt (z.B. \texttt{\dq /de/\dq} für Deutsch, \texttt{\dq /en/\dq} für Englisch, etc.).


\end{document}