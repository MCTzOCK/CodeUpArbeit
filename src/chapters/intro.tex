\documentclass[main.tex]{subfiles}

\begin{document}

\chapter{Einleitung}\label{ch:intro}

CodeUp ist eine Plattform, die Anfängern und besonders jungen Schülern einen einfachen Einstieg in die Programmierung bietet.
In dieser Arbeit wird die zeitliche Entwicklung der Plattform, die technischen Hintergründe und die didaktischen Überlegungen erläutert.
Dabei werden auch auf die Probleme und Herausforderungen berücksichtigt, die während der Entwicklung aufgetreten sind.
Grundsätzlich ist CodeUp in mehrere Bestandteile unterteilt:
\begin{itemize}
    \item Die Webseite, die den Kern des Projekts darstellt und das Hauptinteraktionsmedium für die Nutzer ist.
    \item Die Mobile-App, die es ermöglicht, auch unterwegs zu lernen und zu programmieren.
    \item Die Server-Infrastruktur, die die Kommunikation zwischen den Nutzern und der Plattform ermöglicht.
    \item Die Bereitstellungs-Infrastruktur, die das Veröffentlichen von Projekten und das Arbeiten an Remote-Computern ermöglicht.
\end{itemize}

\section{Motivation}
Die Motiviation für CodeUp kam bei der Suche nach einem geeigneten Thema für ein Jugend forscht Projekt im Jahr 2023.
Bei der Recherche stellte sich heraus, dass es in Deutschland einen erheblichen Fachkräftemangel im IT-Bereich gibt\footnote{\bscite{bitkom-fachkraeftemangel}}.
Dieser Mangel wird sich in den nächsten Jahren noch verschärfen, da immer mehr Unternehmen auf digitale Technologien setzen und entsprechend qualifizierte Mitarbeiter benötigen.
Um diesem Mangel entgegenzuwirken, ist es wichtig, schon frühzeitig junge Menschen für das Programmieren zu begeistern und sie bei der Entwicklung ihrer Fähigkeiten zu unterstützen.

\section{Zielsetzung}
Das Ziel von CodeUp ist es Schülern einen einfachen Einstieg in die Programmierung zu ermöglichen, hierbei wurden folgenden Aspekten
besondere Beachtung geschenkt:
\begin{itemize}
    \item Die Plattform soll so einfach und intuitiv wie möglich zu bedienen sein, damit auch Anfänger ohne Vorkenntnisse schnell erste Erfolge erzielen können.
    \item Es sollen alle benötigten Funktionen und Werkzeuge direkt in die Plattform integriert werden, um den Lernprozess so effizient wie möglich zu gestalten.
    \item CodeUp soll für alle Schüler zugänglich sein, unabhängig von ihrem Standort, ihrer finanziellen Situation oder ihren technischen Vorraussetzungen.
\end{itemize}

\section{Aufbau der Arbeit}
Die Arbeit ist in drei Oberkapitel unterteilt: Webseite, Mobile-App und Infrastruktur.
Dabei behandelt jedes Kapitel einen Teilbereich von CodeUp die zusammen die vollständige Plattform ergeben.
Es wird ein besonderer Fokus auf die technischen Hintergründe gelegt, da diese für das Verständnis der Plattform entscheidend sind.

\end{document}